\chapter{Mao's Inequalities: Proofs and Conjectures}
\section{Mao's conjectures}

Mao, in [18, 19] proved several inequalities using the generating function derived for Dyson's rank on partitions modulo 10 and the $M_2$ rank on partitions without repeated odd parts modulo 6 and 10 including,
$$N(0,10,5n+1)>N(4,10,5n+1)$$
$$N_2(0,6,3n)+N_2(1,6,3n) > N_2(2,6,3n)+ N_2(3,6,3n)$$
Mao also left several inequalities as conjectures for which he had computational evidence.

\newpage
\begin{conjecture}
\begin{equation}
    N(0,10,5n)+N(1,10,5n) > N(4,10,5n)+N(5,10,5n)
\end{equation}
\begin{equation}
    N(1,10,5n)+N(2,10,5n) > N(3,10,5n)+N(4,10,5n)
\end{equation}
\begin{equation}
    N_2(0,10,5n)+N_2(1,10,5n) > N_2(4,10,5n)+N_2(5,10,5n)
\end{equation}
\begin{equation}
    N_2(0,10,5n+4)+N_2(1,10,5n+4) > N_2(4,10,5n+4)+N_2(5,10,5n+4)
\end{equation}
\begin{equation}
    N_2(1,10,5n)+N_2(1,10,5n) > N_2(4,10,5n)+N_2(5,10,5n)
\end{equation}
\begin{equation}
    N_2(1,10,5n+2)+N_2(2,10,5n+2) > N_2(3,10,5n+2)+N_2(4,10,5n+2)
\end{equation}
\begin{equation}
    N_2(0,6,3n+2)+N_2(1,6,3n+2) > N_2(2,6,3n+2)+N_2(3,6,3n+2)
\end{equation}
\end{conjecture}

\subsection{Proofs by Alwaise et al.}

Alwaise et al in [1] gave the proof for (4.1), (4.2), (4.3) and (4.4).
using the generating function of rank differences derived by Mao in [18] and theorems which relies on vector partitions with a given crank.

\newpage
\section{Proof of a limited version of one of Mao's Conjectures}

\subsection{Proof by Barman and Sachdeva}

A limited version of (4.7) was proved by R.Barman using the generating function derived by Mao in [19] and Theorem suggested in [2].

\begin{theorem}
Mao's conjecture (4.7) is true when $3 \nmid n+1$.
\end{theorem}
To prove the above we'll take help of Mao's $M_2$ rank difference generating function.
\begin{theorem}
(Mao [19]). We have 
\begin{equation}
    \begin{split}
        d(n) &:= \sum_{n\geq 0}\big(N_2(0,6,n) + N_2(1,6,n) - N_2(2,6,n) - N_2(3,6,n)\big)q^n \\
        &= \frac{1}{J_{9,36}}\sum_{n = -\infty}^\infty \frac{(-1)^nq^{18n^2 + 9n}}{1+q^{18n+3}} + q\frac{J_{6,36}^2J_{18,36}J_{36}^3}{J_{3,36}^2J_{9,36}J_{15,36}^2}\\
        &+ \frac{J_{6,36}J_{18,36}^2J_{36}^3}{2q J_{9,36}J_{3,36}^2J_{15,36}^2}-\frac{1}{J_{9,36}}\sum_{n = -\infty}^\infty\frac{(-1)^n q^{18n^2+9n-1} }{ 1+q^{18n} }
    \end{split}
\end{equation}
\end{theorem}

But the most important moving part of the proof was due to Barman and Baruah in [2].
\begin{theorem}
We have
\begin{equation}
    \varphi^2(q) + \varphi^2(q^3) = 2\varphi^2(-q^6)\frac{\chi(q)\psi(-q^3)}{\chi(-q)\psi(q^3)}
\end{equation}
\end{theorem}

We look at the generating function $\sum_{n \geq 0}d(3n+2)q^n$ in (4.8).
Now we just take only the exponents congruent to 2 modulo 3 in (4.7) and substitute $q \hookrightarrow q^{\frac{1}{3}}$. We get the following
\begin{proposition}
\begin{equation}
    \sum_{n \geq 0}d(3n+2)q^n  = \frac{1}{qJ_{3,12}}\bigg(\frac{J_{2,12}J_{6,12}^2J_{12}^3}{2J_{1,12}^2J_{5,12}^2} - \sum_{n = -\infty}^\infty \frac{(-1)^n q^{6n^2+3n} }{1+q^{6n} }\bigg)
\end{equation}
\end{proposition}

The following lemma is needed to tie the proof together.
\begin{lemma}
We have
\begin{equation}
    \frac{J_{2,12}J_{6,12}^2J_{12}^3}{J_{1,12}^2J_{5,12}^2} = \frac{\varphi^2(q) + \varphi^2(q^3)}{2}
\end{equation}
\end{lemma}
\begin{proof}
The proof is straight forward manipulation using the properties of q-series and using the Theorem 4.2.3.
\end{proof}
This now sets up to prove Theorem 4.2.1
\begin{proof}
Using Proposition 4.2.4 and Lemma 4.2.5 and noting that all the exponents in the summation inside the parenthesis is $0 \pmod{3}$, we have
\begin{equation}
    \begin{split}
        \sum_{n \geq 0} d(3n+2)q^n  
        &= \frac{ 1 }{ q J_{3,12} } \bigg(\frac{ J_{2,12}J_{6,12}^2J_{12}^3 }{ 2J_{1,12}^2J_{5,12}^2 } - \sum_{n = -\infty}^\infty \frac{(-1)^n q^{6n^2+3n} }{1+q^{6n} }\bigg)
        \\&=\frac{1}{q J_{3,12} } \bigg(\frac{\varphi^2(q) + \varphi^2(q^3)}{4} - \frac{1}{2} + \sum_{n\geq 1}a_{3n}q^{3n}  \bigg)
    \end{split}
\end{equation}
where $a_{3n} \in Z$

Now if $3\nmid n+1$, then
\begin{equation}
    \begin{split}
        d(3n+2)q^n &=[q^n]\frac{1}{q J_{3,12} } \bigg(\frac{\varphi^2(q) + \varphi^2(q^3)}{4} - \frac{1}{2} + \sum_{n\geq 1}a_{3n}q^{3n}  \bigg)
        \\&=[q^{n+1}]\frac{\varphi^2(q) + \varphi^2(q^3)}{4J_{3,12}}
    \end{split}
\end{equation}
where $[x^k]f(x)$ denotes the coefficient of $x^k$ in $f(x)$. Now all that we need to show is that all the coefficients of $\frac{\varphi^2(q) + \varphi^2(q^3)}{4J_{3,12}}$ are positive. This can be done using the following simplification
\begin{equation*}
    \begin{split}
        \frac{\varphi^2(q) + \varphi^2(q^3)}{4J_{3,12}} &= 
        \frac{2+4q+4q^2 + \sum_{n\geq 3}b_nq^n   }{(1-q^3)(q^9,q^9,q^12;q^12)_\infty}
        \\&= \bigg( 2+4q+4q^2 + \sum_{n\geq 3}b_nq^n \bigg) \bigg(\sum_{n\geq0}q^{3n}\bigg)\bigg(1+\sum_{n\geq0}c_nq^n\bigg)
    \end{split}
\end{equation*}
where $b_i$ and $c_i$ are non-negative.
\end{proof}
\subsection{Remarks}

The result is limited to $3n+2$ when $3\nmid n$ but computational evidence is suggested in [3] for the remaining case too.

To prove the result for the remainder of the integers would imply that the following expression has non-negative coefficients
$$\frac{1}{1-q^{12}}\bigg(\frac{J_{2,12}J_{6,12}^2J_{12}^3}{2J_{1,12}^2J_{5,12}^2} - \sum_{n = -\infty}^{n = \infty}\frac{(-1)^nq^{6n^2+3n}}{1+q^{6n}}\bigg)$$

which would also mean that it is equivalent to showing that the following has positive coefficients for all exponents of $q$.
$$\frac{1}{qJ_{3,12}}\bigg(\frac{\varphi^2(q) + \varphi^2(q^3)}{4} - \frac{1}{2} + \sum_{n\geq 1}a_{3n}q^{3n}\bigg)$$

Our first attempt was to use the properties of $\varphi^2(q)$, where the coefficient of $q^n$ counts the number of integer solutions to $a^2 + b^2 = n$ but we need to compare this number with the other half of the above expression which does not seem to have any nice properties like this.

We could attempt a proof along the lines of the method used by Alwaise et. al. in [1] but in the general setting removing a factor with non-negative coefficient can yield a negative coefficient which is hard to convince against. 

