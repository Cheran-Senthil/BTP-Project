\chapter{Introduction to partitions}



\section{Background}

A partition of a positive integer $n$ is a way of writing $n$ as the sum of positive integers. Two sums that differ only in the order of their summands are considered the same partition. The notation $\lambda \vdash n$ means that $\lambda$ is a partition of $n$. 

A summand in a partition is also called a part. The length of a partition refers to the number of parts in the partition and is denoted by $n(\lambda)$ for a partition $\lambda$. Its largest part, is denoted by $l(\lambda)$. As an example, 4 can be partitions in five distinct ways as follows
\begin{align*}
4 &= 4
  \\&= 3 + 1 = 2 + 2 
  \\&= 2 + 1 + 1 
  \\&= 1 + 1 + 1 + 1.
\end{align*}

Where, $n((2,1,1)) = 3$ and $l((2,1,1)) = 2$.
\newpage

\section{Definitions and Terminology}
\subsection{$q$-Pochhammer Symbol}

The $q$-Pochhammer symbol is defined as

\begin{equation*}
    (a)_0 := (a;q)_0 := 1, \quad (a; q)_n := \prod_{k=0}^{n-1} (1- aq^k) \quad n \geq 1,
\end{equation*}
\begin{equation*}
    (a)_\infty := (a;q)_\infty := \prod_{k=0}^{\infty} (1- aq^k) \quad |q| < 1.
\end{equation*}
If the identification of $q$ is clear, we omit $q$ from the notation.
We state several basic identities used for simplification later as mentioned in [5].
\begin{equation}
    (q^a;q^b)_\infty(-q^a;q^b)_\infty = (q^{2a};q^{2b})_\infty
\end{equation}
\begin{equation}
    (cq^a;q^{2b})_\infty(cq^{a+b};q^{2b})_\infty = (cq^a;q^b)_\infty
\end{equation}
We will be using the following shorthand notations throughout the report and are also used in [1, 18, 19]:
\begin{equation*}
    (a_1,...,a_k;q)_n := (a_1,q)_n...(a_k;q)_n
\end{equation*}
\begin{equation*}
    J_b := (q^b;q^b)_\infty
\end{equation*}
\begin{equation*}
    J_{a,b} := (q^a,q^{b-a},q^b;q^b)_\infty
\end{equation*}
\begin{equation*}
    \chi(q) := (-q;q^2)_\infty
\end{equation*}
\subsection{Ramanujan's General Theta Function}
The famous Ramanujan's Theta function used in various proofs is as follows:
\begin{equation}
    f(a, b) := \sum_{n = -\infty}^\infty a^{n(n+1)/2}b^{n(n-1)/2}, \quad |ab| < 1.
\end{equation}
Some shorthand related to the above generating functions will be used as [5]
\begin{align*}
    \varphi(q) &:= f(q,q) = (-q,-q,q^2;q^2)_\infty,
    \\ \psi(q) &:= f(q,q^3) = \frac{(q^2;q^2)_\infty}{(q;q^2)_\infty}.
\end{align*}

\subsection{Definitions Relating to Partitions}
\subsubsection{Partition Functions}
If $n$ is a positive integer, let $p(n)$ denote the number of unrestricted representations of $n$ as a sum of positive integers, where representations with different orders of the same summands are not regarded as distinct. We call $p(n)$ the partition function. We further use the notation $p_m(n)$ to denote the number of partitions of $n$ into parts that are no larger than $m$.

In general for a set $S$, $p(S, m, n)$ denotes the number of partitions of $n$ into exactly $m$ parts of $S$. Furthermore, we denote $p(m,n)$ as the number of partitions of $n$ into exactly $m$ parts.

We denote $Q(n)$ as the number of partitions into distinct parts. More generally, $Q( S, m, n)$ denotes the number of partitions of $n$ into $m$ distinct parts of $S$. Similarly $Q( m, n )$ is the number of partitions of $n$ into exactly $m$ distinct parts.

\subsubsection{Overpartitions}
An overpartition of $n$ is a partition of $n$ in which the first occurrence of a number may be over-lined. For example, the over partitions of 3 are:
\begin{align*}
      \{ 3, \overline 3
    , 2 + 1 , 2 + \overline 1 , \overline 2 + 1 , \overline 2 + \overline 1  
    , 1 + 1 + 1 , \overline 1 + 1 + 1 \}
\end{align*}

The number of overpartitions of $n$ is denoted by $\overline p(n)$. This results in
    $$\overline p(n) = \sum_{n_1 + n _2 = n} p(n_1)*Q(n_2)$$

which is a discrete convolution between $p$ and $Q$, giving us $\overline p = p*Q$. Analogous to Dyson's rank function definitions for partitions we have $\overline N(s, n)$ and $\overline N(s, m, n)$ for overpartitions.

\subsection{Generating Functions}
The term generating function is used to describe an infinite sequence of numbers $(a_n)$ by treating them as the coefficients of a series expansion. This infinite series is the generating function.

The generating function for the partition function $p(n)$ is represented as
\begin{align*}
\sum_{n=0}^\infty p(n)q^n
&= \sum_{n_i \geq 0}q^{1n_1 + 2n_2 + \dots + kn_k + \dots}\\
&= \frac{1}{1-q} \frac{1}{1-q^2} \dots \frac{1}{1-q^k} \dots\\
&= \frac{1}{\prod_{k=0}^\infty (1-qq^k)}\\
&= \frac{1}{(q;q)_\infty}
\end{align*}
\newpage
For general partition functions, we get
\begin{align*}
\sum_{n=0}^\infty \sum_{m=0}^\infty p( S, m, n )z^mq^n &= \prod_{k \epsilon S} \frac{1}{1-zq^k},\\
\sum_{n=0}^\infty \sum_{m=0}^\infty Q( S, m, n )z^mq^n &= \prod_{k \epsilon S} (1+zq^k).
\end{align*}

The generating function for overpartitions,  $\sum_{n=0}^\infty \overline p(n)q^n$ is the product of general partitions and that for partitions with distinct parts as it is given by their convolution. Therefore,
\begin{align*}
\sum_{n=0}^\infty \overline p(n)q^n = \frac{(-q;q)_\infty}{(q;q)_\infty}
\end{align*}


